\section{Achieved Results}

	In this writing, the tool \emph{q2d}, its theoretical background and design were presented.
	
	\paragraph{Goal}
	The intention behind this work was to create an application that allows users to easily create a circuit design containing configurable components within a graphical \gls{ui} along with a specification of an intended behaviour.
	The tool evaluates whether this specification can be fulfilled by a component configuration and, if this is the case, also provides a configuration that solves the posed problem.
	It was established that this requires the solving of a \gls{qbf} problem and how the design can be described as such.
	
	\paragraph{Development and Design Decisions}
	Initially, it was intended to use \gls{qucs} as a code base and employ a pure \gls{sat} solver for evaluating the designed circuits.
	This approach turned out to be infeasible due to design issues and the state of \gls{qucs}' code at the time of writing.
	As a consequence, it was chosen to implement a new tool for the intended purpose.
	 From the experience made with the earlier attempt, several major design decisions were inferred:
	 
	 \begin{itemize}
	 
	 \item The application was developed using \gls{qt}.
	 	Retrospectively, the time invested in getting familiar with this framework seemed to be well invested due to the amount of work saved when dealing with common implementation tasks.
	 	
	 \item The \gls{qbf} solver \gls{quantor} has been employed instead of a pure \gls{sat} solver.
	 	This eradicated the need for intermediate file formats and streamlined the circuit evaluation process considerably.
	 
	 \item Components are not hard-coded.
	 	They instead reside in separate descriptor files, which are easy to write, read and adapt by users.
	 	
	 \item Component behaviour is described using boolean formulae or \gls{cnf}.
	 The same applies to the description of the intended functionality of the circuit.
	 Allowing different variations of specifying operators allows users to use the notation they prefer.
	 		 	
	 \item Automatically generated circuit symbols for components relief the users of providing such for each component type they design.
	 As a result, custom components can be developed even quicker.
	 
	 \end{itemize}
	
	\paragraph{Functionality and Appearance}
	Basic project and document management features are available and users are enabled to import and employ any component they have a descriptor file for.
	Several visualisation techniques have been employed to allow a faster orientation within the schematic design and improve the perception of relevant information.
	An emphasis has been put on the presentation of component ports.
	Additionally, \emph{details on demand} are available for all circuit elements.
	
	\paragraph{Implementation Details}
	The core classes of the application and their interaction with each other have been described.
	The focus mostly lies on their purpose and concepts.
	For the \texttt{QuantorInterface} the inner workings have been further elaborated . 
	It thereby was underlined how the theoretical approach was realized in the application.
	
	\paragraph{A Workflow Example}
	To demonstrate the usage of the tool, the schematic of a \emph{Xilinx Virtex 5} \gls{clb} has been reproduced, implementing the required custom components in the process.
	 It was shown that a \emph{full adder} could be implemented with such a \gls{clb}, by computing a configuration that solved the posed problem.