\section{The Document Entry Class and its Factory}
	\label{sec-documentEntry}
	Section  \ref{sec-basic-interaction} established the management of data in \emph{documents}.
	Since each of these documents describes a circuit as a whole, it is obvious that a data structure must exist, that can represent the elements of which the circuit is constructed.
	This is achieved by introducing \emph{document entries}.
	A \gls{uml}-representation of the classes discussed in the following is given in figure \ref{fig-class-diagram-documentEntry}.	
	
	\paragraph{The Environment of Document Entries}
	Any document entry needs to be associated with exactly one \emph{document}, which in turn might contain an arbitrary number of entries.
	Amongst each other, entries that belong to the same document can form hierarchies.
	Like all classes in the application, that might be involved in such a parent-child relationship, the \texttt{DocumentEntry} class inherits from \texttt{QObject}.
	As an additional feature, this also provides access to \gls{qt}s \emph{signal-slot} syntax, reducing the effort of implementing the communication between class instances notably.
	Internally, the application is split into a \emph{model}, holding the circuits structure and a \emph{schematic} representation, used by the \gls{ui}.
	A document entry acts as a controller to the two, subjecting them to changes depending on the operations issued by the user.
	It therefore is also called a \emph{related entry}.
	Since depending on the underlying model- and schematic elements, different interactions are possible, each document entry possesses a type which indicates what kind of elements it represents.\footnote{
	Subclassing \texttt{DocumentEntry} also would have been a viable	option.
	The projected gains however did not accommodate for the required efforts during the initial design.
	As always, this might change in the future, should the need arise.
	}
	This subdivision is discussed further in section \ref{sec-model-view}.
	Document entries are aware of the overall \emph{model} and \emph{schematic} indirectly through their parent document, and offer accessors to them for convenience reasons. 

	
	\paragraph{The \texttt{Identifiable} Helper Class}
	From the hierarchical structure of documents and their entries, the need to identify them by a name arose.
	For this reason, a helper class was developed, providing an abstraction for classes that require their instances to be recognized by a unique token.
	Upon the generation of such an identifiable object, it is only required to provide a \emph{local name}, and an optional parent object.
	Should the latter not be given, the current object may serve as root of a new hierarchy.
	A so-called \emph{full ID} consists of a local ID prepended with the name of each parent up to the hierarchies root.
	
	To aid with conceptualization, it is helpful to think of a file system, where the local name is only the file or folder name per\-se and the full ID can be compared to an absolute path.
	
	A useful result of this approach is the possibility to query any given document for an entries full ID and obtain the corresponding object, if there is one.
	Also, the \texttt{Identifiable} class creates a central point of validation for all IDs used throughout the application.	
	
	\paragraph{The Document Entry Factory}
	Instantiating document entries requires a lot of secondary tasks, such as ID validation, construction of model- and schematic elements, bookkeeping.
	To ease the effort and reduce the likelihood of errors, a factory class has been introduced, which takes care of all the matters that are associated with instantiating a \texttt{DocumentEntry}.
	For this purpose, a set of functions is offered to create entries for each possible type.
	Since ports are always sub-entries to some component or module interface\footnote{
		Module interfaces have been introduced in section \ref{sec-visualization-aspects}.	
	}
	, they are not instanced independently and therefore the corresponding helper function was omitted here.
	The default value for any of the IDs is invalid, triggering the factory to generate a valid one as a replacement.
	Since the user usually does not want to bother with naming everything in advance, this behaviour was deemed desirable.
	
	