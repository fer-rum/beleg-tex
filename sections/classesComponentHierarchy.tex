\section{Classes Involved in the Component Meta-Model}
	\label{sec-metamodel}
	The \emph{component meta-model} has been established in section \ref{sec-model-design}.
	Its implementation can be found in the \texttt{q2d::metamodel}-namespace.
	Each part of the class hierarchy inherits from \gls{qt}'s \texttt{QObject} and \texttt{QStandardtem}.
	This is mainly to make use of the model-view capabilities delivered by the framework and ease the representation of the hierarchy in the \gls{ui} considerably.

	\paragraph{Overview over Elements}
	Everything in the meta-model is considered an \emph{element} of the model.
	Elements possess a \emph{type}, a \emph{name}, and may have another element as \emph{parent}.
	The type may be one of those listed in the \emph{element type} enumeration.
	Setting the name is achieved by using the \texttt{QStandardItem} constructor, while the parent relationship is inherited from \texttt{QObject}, narrowing down the type.
	
	The treelike structure of the meta-model is created by the subclass of \emph{hierarchy elements}.
	Additionally to their name, they also have a \emph{hierarchy name}, which serves as unique identifier, since it includes the names of all parent elements in addition to the elements own name.
	The principle is similar to absolute path names used in \glspl{os}.\footnote{
		Later during the implementation, a comparable naming and identification system for component instances had been introduced.
		Since both systems are very alike it is possible, that they will be merged eventually.
		The \texttt{Identifiable} class is discussed in section \ref{sec-documentEntry}.
	}
	Further, hierarchy elements are restricted to only accept \emph{categories} as parents.
	The intention is, to avoid that component types contain other component types, or even themselves.
	A nesting of categories on the other side poses no problem.
	
	Since \emph{component types} are the focus of the meta-model, it is obvious, to have a subclass for \emph{component elements}, from which component types can then be composed.
	Contrary to their superclass, these are restricted to component types as parent elements. 
	
	\paragraph{Categories}
	To keep an overview over many component types, categories have been introduced.
	They may contain an arbitrary set of \emph{component descriptors} or other sub-categories.
	In the \gls{ui} representation, categories can be expanded and collapsed by the user to help maintaining an overview or finding certain elements.
	
	\paragraph{Component Descriptors}
	As motivated in section \ref{sec-model-design}, an abstraction into component types was desired.
	This task is fulfilled by the \texttt{ComponentDescriptor}-class.
	It acts the as parent element for everything that is needed to describe a component type's behaviour and available ports.
	Further, information about the visual representation of a component are stored.
	Instances of this class are derived from the user definitions as discussed in section \ref{sec-user-defined-components}.
	
	\paragraph{Port-, Function-, and Configuration Bit Group Descriptors}
	When creating a component type, sub-descriptors are required.
	These are also provided by the descriptor files.
	Connection facilities are covered by \emph{port descriptors}, specifying their direction and the position relative to the component symbols origin.
	
	A \gls{pcbd}, as established in sections \ref{sec-definitions} and \ref{sec-functional-behaviour-spec}, is represented by a \emph{function descriptor}.\footnote{
		Careful readers might note that \glspl{pcbd} are, strictly speaking, no functions at all.
		This became clear during the work on the assignment, the name stuck however.
	}
	The fact that \glspl{pcbd} are given as strings, is reflected in the rather simple constructor.

	Additionally, component types may include \emph{configuration bit groups}.
	These are given by a name and a number of members, and upon adding them to a \texttt{ComponentDescriptor}-instance, will be expanded into a set of separate \emph{configuration bits}.
	
	\paragraph{Configuration Bit Descriptors}
	Contrary to the other parts of a component descriptor, these are not explicitly described in the respective file but are created by \emph{q2d} to have representative elements for each configuration bit in the component.
	This becomes useful when generating the solution to the \gls{qbf} problem.
	
	
	