\section{Theoretical Background}
	\label{sec-theoretical-background}
	
	When looking at the posed problem from a mathematical perspective, it becomes clear that the solving of a \gls{qbf} problem is required.
	Intuitively speaking, the designed circuit and its components are to be described by boolean formulae. 
	For those, an interpretation is to be found, which lets all formulae become true.
	As such a satisfying interpretation defines suitable values for the configurable circuit elements, it will be called a \emph{configuration}.

\subsection{Definitions}
	\label{sec-definitions}
	
	For simplicity, a shorthand notation for the set of boolean values will be used as provided in equation \ref{eqn-define-Bool}.
	
	\begin{equation}
		\label{eqn-define-Bool}
		\mathbb{B} := \lbrace \bot, \top \rbrace
	\end{equation}

	When talking about circuits, the need to distinguish its elements from each other arises, since each of them establishes its own \emph{context}.
	This will be achieved by marking symbols referring to circuit element $\epsilon$ with the index $\epsilon$.
	If no index is given, the symbol refers to the circuit as a whole.
	This will also be referred to as the \emph{global context}.

	Further, it is useful to define three disjunct sets of variables. 
	Each of these sets has a distinct semantic meaning.
	
	\begin{description}
		\item[$C_\epsilon$] Configuration variables represent programmable bits that define the temporary specific behaviour of an element.
		
		\item[$N_\epsilon$] Node variables represent the interfacing points of components that serve as ports for interconnecting the overall circuit.
		
		\item[$X$] Input variables stand for the outside connections of the whole circuit that are driven arbitrarily during the circuit's operation by the surrounding world.
	\end{description}
	
	Let a \gls{pcbd} be given as a boolean formula on a set of configuration and node variables as in equation \ref{eqn-define-fsmall}.

	\begin{equation}
		\label{eqn-define-fsmall}
			f_{\epsilon, i}\left( V_\epsilon \right)
		\left|
			\begin{array}{l}
			V_\epsilon \subseteq \left( C_\epsilon \cup N_\epsilon \right) \\
			i \in \left[0,n\right); \, n \in \mathbb{N}
			\end{array}
		\right.
	\end{equation}

	An arbitrary large set of \glspl{pcbd} can be used to describe the behaviour of a single circuit element such as a component or a wiring.
	All those \glspl{pcbd} are combined in equation \ref{eqn-define-F-epsilon} into a set describing the whole circuit elements behaviour, called the \gls{cbd}.
	
	\begin{equation}
		\label{eqn-define-F-epsilon}
			F_\epsilon := \bigcup_n f_{\epsilon, n} \left( V_\epsilon \right)
	\end{equation}	
	
	% N, C, F form context of \epsilon	
	For any given element $\epsilon$, its \emph{local context} is defined in equation \ref{eqn-define-box-epsilon} as
	
	\begin{equation}
		\label{eqn-define-box-epsilon}
		\Box_\epsilon := \left(C_\epsilon, N_\epsilon, F_\epsilon \right)
	\end{equation}
	
\subsection{Expressing Connections between Circuit Elements}
	In the case of node variables, it may occur that two elements effectively refer to the same node.\footnote{
		For example, one might think of a wire that is connected to the port of a component.	
	}
	Then, given the existence of two elements $\epsilon_a$ and $\epsilon_b$ sharing the same node, this common node can be expressed as a boolean formula $\sigma_{a, b}$. 
	
	\begin{equation}
	\sigma_{a, b} = 
	\left\lbrace
	n_a \Leftrightarrow n_b
	\vert
		n_a \in N_{\epsilon_a} \wedge 
		n_b \in N_{\epsilon_b} \wedge 
		\, \text{$n_a$ and $n_b$ are connected}
	\right\rbrace
	\label{eqn-define-sigma}
	\end{equation}
	
	Note that the equivalence can be transformed as shown in equation \ref{eqn-resolve-equivalence}. 

	\begin{equation}
	\begin{aligned}
		x \Leftrightarrow y 
		&\equiv \left( x \Leftarrow y \right) \wedge \left( x \Rightarrow y \right) \\
		&\equiv \left( \neg x \vee y \right) \wedge \left( x \vee \neg y \right) \\
		&\equiv \left[ \bar{x}, y \right] \left[ x, \bar{y} \right]
	\end{aligned}
	\label{eqn-resolve-equivalence}
	\end{equation}
	This will come in handy at several points in the future, especially when discussing conversions into \gls{cnf}.
	
\subsection{Global Context and Target Function}

	\paragraph{Creating the Global Context}
	To obtain a holistic description of the circuit, it is required to create a global context for the whole described circuit.
	For this purpose, the sets of node and configuration variables are merged as shown in equation \ref{eqn-merged-varsets}.
	
	\begin{equation}
	\begin{aligned}
		N &:= \bigcup_\epsilon N_\epsilon \\
		C &:= \bigcup_\epsilon C_\epsilon
	\end{aligned}
	\label{eqn-merged-varsets}
	\end{equation}
	
	In the global context, there also exist input variables that have to be mapped to node variables to express how the inputs are connected to the circuit.
	This mapping is quite similar to the definition of the $\sigma$-notation in equation \ref{eqn-define-sigma} and consequently is called $\sigma^*$.
	
	\begin{equation}
	\sigma^* = 
	\left\lbrace	
		x \Leftrightarrow n \\
	\vert
		x \in X \wedge
		n \in N \wedge
		\, \text{$x$ and $n$ are connected}
	\right\rbrace
	\label{eqn-define-sigma-star}
	\end{equation}
	
	One might wish to ensure that all inputs to a circuit are connected by requiring
	\begin{equation}
		\forall x. \exists n.
		\left( 
			x \Leftrightarrow n
		\right) \in \sigma^*\left( x, n \right) 
	\left| 
	\begin{array}{l}
		x \in X \\
		n \in N
	\end{array}
	\right.
	\end{equation}		
	
	This, however is not necessary for the problem to be decidable and will not be enforced.
	As a result, a global set of formulae set $F$ can be created as
	
	\begin{equation}
		F := 
		\bigcup_{\epsilon} F_\epsilon \cup
		\bigcup_{a, b} \sigma_{a, b} \cup 
		\sigma^*
	\end{equation} 	
	
	This can also be merged into a single formula $F^*$ by conjugating all elements of $F$.\footnote{
		One may find that node variables could be removed at this point by merging functions into each other and creating one large transfer formula that describes the whole circuit.
		There are, however, no practical gains proceeding so since the automated solving of the problem might re-introduce them eventually.
		Also, solvers are capable of performing clause and literal eliminations by themselves.
	}	
	
	\begin{equation}
		F^* := \bigwedge_{f \in F} f
	\end{equation}	
	
	Comparable boolean description techniques of circuit behaviour can be found in \cite{EfficientSAT} and \cite{FPGA_PLB}.
	
	Conclusively, the global context can now be represented by 

	\begin{equation}
		\Box := \left(C, N, X, F \right)
		\label{eqn-define-box}
	\end{equation}	

	\paragraph{Including the Target Formula}
	Once all contexts are established, the last thing left to specify is the \emph{target formula}.
	It describes desired behaviour of the given circuit partially.\footnote{
		Given the case that there exists only one partial target formula it trivially equals the complete target formula.
	}
	For that purpose, an indexed target formula $t_i$ can be defined as  
	
	\begin{equation}
	t_ i\left( V \right)
	\left|
	\begin{array}{l}
		V \subseteq \left( X \cup N \right) \\
		i \in \left[0,n\right); \, n \in \mathbb{N}
	\end{array}
	\right.
	\end{equation}

	To maintain a consistency with the ways of defining and naming things, the set of partial target functions is called $T$.

	\begin{equation}
		T := \bigcup_i t_i
	\left|
		i \in \mathbb{N}
	\right.
	\end{equation}		
	
	Since all target functions need to be fulfilled to solve the problem, it is an obvious step to conjunct them into a general target function $T^*$.
	
	\begin{equation}
		T^* := \bigwedge_{t \in T} t
	\end{equation}	 
	
\subsection{Problem formulation as QBF and SAT}
	\label{sec-qbf-to-sat}
	With all prerequisites established, equation \ref{eqn-qbf-representation} then encodes the fact that there exists a configuration that for all input combinations, there is a node assignment so that $F^* \wedge T^*$ is fulfilled.

	\begin{equation}
		\label{eqn-qbf-representation}
			\exists c.\forall x. \exists n. \left( F^* \wedge T^* \right)
		\left|
		\begin{array}{l}		 
			c \in C;\ x \in X;\  n \in N
		\end{array}	
		\right.
	\end{equation}
	
	This formula establishes an \emph{EAE-problem}, which is a special instance of a \gls{qbf}.
	Many tools require a representation in \gls{cnf}.
	It is possible to translate \gls{qbf}-problems into \gls{sat}-problems although the sets of clauses and variables will expand largely. \cite{SATinQBF}
	
	Section \ref{sec-solving-qbf} will discuss how this problem has been approached in the actual implementation.
	
	\paragraph{Looking for Multiple Solutions}
	Once the problem formulation is present in \gls{cnf}, it can be evaluated, whether the problem is satisfiable.
	Should this be the case, the negated solution can be added as a clause to the problem.
	If this also turns out to be satisfiable, an alternative solution has been found.
	This process can be repeated until the problem is no longer satisfiable, at which point all possible solutions have been found.\footnote{
		In the actual implementation, the \gls{sat} solver might abort not only due to the problem being not satisfiable but also when running out of memory or exceeding a certain computation time.
		Usually solvers report the reason for stopping early.
		Still, only non-satisfiability can definitively negate the existence of further solutions.
	}