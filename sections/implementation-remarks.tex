In this section, implementation details of the application will be discussed.
Since the actual source files contain several thousands line of code, as well as dozens of classes and methods, it is not viable to explain everything in detail.
Therefore, several simplifications have been made in the following, to maintain the focus on the presented features.

\begin{itemize}
	\item All classes and methods that belong to the tool's implementation always reside in the super-namespace \texttt{q2d}, even if not explicitly stated.
	
	\item Classes that originate from \gls{qt} are not explicitly marked as such.
	They still can be easily spotted since it is a convention within the framework that all class names are prefixed with an uppercase letter \emph{Q}.
	\item The presentation has been reduced to focus on the main structure.
	For this reason, accessors, attributes and methods used for bookkeeping or as helpers have been omitted unless they are in some way important for explanatory reasons.
	
	\item Code elements, that were already considered deprecated at the time of writing, also have been left out.
\end{itemize}

Any description given here naturally can only reflect the application's state at a certain point in time.
To obtain further information on implementation details, it is strongly recommended to check out the project repository from \gls{github}.\footnote{
	\url{github.com/fer-rum/q2d}
}
In addition to the documentation delivered with the source code, a wiki is available at this site to help answering questions the user might have.\footnote{
	Should the wiki not be able to provide a satisfying answer, an issue should be opened, which will cause the developers to expand the wiki to cover the posed question as well. 
}

\Gls{uml} diagrams have been added to this document to aid understanding.
They can be found in appendix~\ref{sec-appendix-uml}