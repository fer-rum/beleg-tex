

\chapter{UML Diagrams}
	\label{sec-appendix-uml}
	The following \gls{uml} diagrams are referenced and explained in chapter \ref{sec-chapter-implementation}.
	It is recommended to read it beforehand or in parallel to inspecting the diagrams to facilitate understanding the topic.
	
	\paragraph{Qt classes}
	At the time of writing, \gls{qt} version 5.4 was used in the implementation.
	The official documentation for these classes can be found at \url{doc.qt.io/qt-5/classes.html}
	
\pagebreak


	\begin{sidewaysfigure}
		\centering
		\resizebox{\textheight}{!}{
		\begin{tikzpicture}

\draw[draw=none] (0,0) rectangle (52, 34);

% -- Qt classes
\umlclass[x=18, y=28]{QObject}{}{
+ QObject(parent : QObject*)
}

\umlclass[x=34, y=28]{QStandardItem}{}{
+ QStandardItem(text : QString) \\
\umlvirt{+ type() : int}
}

\begin{umlpackage}{q2d}
	\begin{umlpackage}{metamodel}

		\begin{umlpackage}{enums}
		\umlclass[type=enum, x=10, y=20]{ElementType}{
			CATEGORY \\
			COMPONENT \\
			CONFIG\_BIT \\
			CONFIG\_BIT\-GROUP \\
			FUNCTION \\
			INVALID \\
			PORT \\
		}{}
		\end{umlpackage} % --- enums

	\umlclass[x=26, y=20, type=abstract]{Element}{
	}{
	+ Element(name : QString, parent : Element*) \\
	\umlvirt{+ type() : int}
	}
	\umldep{Element}{ElementType}
	\umlinherit[geometry=|-]{Element}{QObject}	
	\umlinherit[geometry=|-]{Element}{QStandardItem}

	\umlclass[x=13, y=14, type=abstract]{HierarchyElement}{
	-hierarchyName : QString
	}{
	+ HierarchyElement(name : QString, parent : Category*) \\
	}
	\umlVHVinherit{HierarchyElement}{Element}

	\umlclass[x=34, y=14, type=abstract]{ComponentElement}{ % --- ComponentElement
	-hierarchyName : QString
	}{
	+ ComponentElement(name : QString, parent : ComponentDescriptor*) \\
	}
	\umlVHVinherit{ComponentElement}{Element}

	\umlclass[x=5, y= 6]{Category}{}{ % --- Category
	+ addSubItem(toAdd : HierarcyElement) \\
	+ type() : int
	}
	\umlinherit[geometry=|-|, anchors = north and south]{Category}{HierarchyElement}
	\umlunicompo
	[arg1=parent, mult1=*, pos2=1.8, mult2=1, geometry=-| , anchors=west and 130]
	{HierarchyElement}{Category}

	\umlclass[x=16, y=6]{ComponentDescriptor}{}{ % --- ComponentDescriptor
	+ addConfigBitGroup(configBitGroup : ConfigBitGroupDescriptor*) \\
	+ addFunction(function : QString) \\
	+ addPort(port : PortDescriptor*) \\
	+ setSymbolPath(path : QString) \\
	+ type() : int
	}
	\umlVHVinherit[anchors= north and south]{ComponentDescriptor}{HierarchyElement}
	\umlunicompo
		[arg1=parent, pos1=0.1, align1=right, mult1=*, 
		pos2=1.8, align2=left, mult2=1, geometry=-| , anchors=west and 30]
		{ComponentElement}{ComponentDescriptor}

	\umlclass[x=30, y=6]{ConfigBitGroupDescriptor}{- memberCount : unsigned}{ % --- ConfigBitGroupDescriptor
	+ ConfigBitGroupDescriptor(name : QString, memberCount : unsigned, $\hookleftarrow$ \\
	\hspace{1cm}parent : ComponentDescriptor*) \\
	+ type() : int
	}
	\umlVHVinherit[anchors= north and south]
	{ConfigBitGroupDescriptor}{ComponentElement}

	\umlclass[x=25, y=10]{ConfigBitDescriptor}{}{ % --- ConfigBitDescriptor
	+ type() : int
	}
	\umlHVinherit[anchors= east and south]
	{ConfigBitDescriptor}{ComponentElement}
	\umldep
	[stereo=generate, pos stereo = 2.5,  anchors = 150 and south, geometry=|-|]
	{ConfigBitGroupDescriptor}{ConfigBitDescriptor}

	\umlclass[x=44, y=6]{FunctionDescriptor}{}{ % --- FunctionDescriptor
	+ FunctionDescriptor(function : QString, parent : ComponentDescriptor*) \\
	+ type() : int
	}
	\umlVHVinherit[anchors= north and south]
	{FunctionDescriptor}{ComponentElement}

	\umlclass[x=48, y=10]{PortDescriptor}{ % --- PortDiscriptor
	- direction : PortDirection
	}{
	+ position() : QPoint \\
	+ type() : int
	}
	\umlHVinherit[anchors= west and south]
	{PortDescriptor}{ComponentElement}	

	\end{umlpackage} % --- metamodel
\end{umlpackage} 

\umlclass[x =48, y=28, type=enum]{PortDirection}{
			IN \\
			OUT
		}{}
\umlunicompo[anchors= north and south, geometry=|-|]{PortDescriptor}{PortDirection}
\umlnote[x=42, y=28]{PortDirection}{namespace q2d::model::enums}

\end{tikzpicture} 
		}
		\caption{\gls{uml}-Diagram for the Component Meta-Model}
		\label{fig-class-diagram-metamodel}
	\end{sidewaysfigure}

	\begin{sidewaysfigure}
		\centering
		\resizebox{\textheight}{!}{
		\begin{tikzpicture}

\draw[draw=none](0,0) rectangle (52, 34);

\umlclass[x=26, y= 30]{QObject}{}{
+ QObject(parent : QObject*)
}

\begin{umlpackage}{q2d}
	\umlclass[x=26, y=17]{DocumentEntry}{
	}{
	+ DocumentEnty(id : QString, type : DocumentEntryType, $\hookleftarrow$ \\
	\hspace{1cm}document : Document*, parent : DocumentEntry*) \\
	+ setModelElement(modelElement : ModelElement*) \\
	+ setSchematicElement(schematicElement : SchematicElement*) \\
	+ document() : Document* \\
	+ model() : Model \\
	+ schematic() : Schematic \\
	}

	\umlinherit
	[anchors= north and south]
	{DocumentEntry}{QObject}

	\umlclass[x=6, y=17]{Document}{}{
	+ addEntry(entry : DocumentEntry*) \\
	+ entryForFullId(id : QString) : DocumentEntry*
	}

	\umlunicompo
	[anchor1=170, anchor2=-20, geometry=-|,
	arg2= document, mult2=1, mult1=1]
	{DocumentEntry}{Document}

	\umluniaggreg
	[anchor1=east, anchor2=west,
	arg2=entries, mult2=*, mult1=1]
	{Document}{DocumentEntry}

	\umlinherit
	[anchors= north and west, geometry=|-]
	{Document}{QObject}

	\begin{umlpackage}{model}
		\umlclass[x=6, y=10]{ModelElement}{}{}
		\umlclass[x=16, y=10]{Model}{}{}
		\umluniaggreg{Model}{ModelElement}
	\end{umlpackage} % --- model

	\umlVHVunicompo
	[anchors= -130 and north,
	mult1=1, mult2=1, pos2=2.5, arg2 = modelElement]
	{DocumentEntry}{ModelElement}

	\umlVHdep{DocumentEntry}{Model}

	\begin{umlpackage}{gui}
		\umlclass[x=46, y=25]{SchematicElement}{}{}
		\umlclass[x=36, y=25]{Schematic}{}{}
		\umluniaggreg{Schematic}{SchematicElement}
	\end{umlpackage} % --- gui

	\umlVHVunicompo
	[anchors= -50 and north,
	mult1=1, mult2=1, pos2=2.5, arg2 = schematicElement]
	{DocumentEntry}{SchematicElement}

	\umlVHdep[anchor1=70]{DocumentEntry}{Schematic}

	\begin{umlpackage}{core}
		\umlclass[x=16, y=25]{Identifiable}{}{
		+ Identifiable(localId : QString, parent : Identifiable*) \\
		+ localId() : QString \\
		+ fullId() : QString \\
		\umlstatic{+ isValidLocalId(id : QString) : bool} \\
		\umlstatic{+ isValidFullId(id : QString) : bool} \\
		}

		\umlunicompo[anchors = south and 150, geometry=|-|, arm1=-2cm, arm2=-cm, 
		arg2 = parent, mult2= 0..1, mult1=1, pos2=2.25]
		{Identifiable}{Identifiable}
	\end{umlpackage} % --- core

	\umlinherit
	[anchors=north and east, geometry=|-]
	{DocumentEntry}{Identifiable}

	\umlinherit
	[anchors=north and west, geometry=|-]
	{Document}{Identifiable}


	\begin{umlpackage}{enums}
		\umlclass[x=41, y=17, type=enum]{DocumentEntryType}{
			COMPONENT \\
			MODULE\_INTERFACE \\
			PORT \\
			WIRE
		}{}
	\end{umlpackage} % --- enums

	\umlunicompo
	[mult1=1, mult2=1, arg2=type]
	{DocumentEntry}{DocumentEntryType}

	\begin{umlpackage}{factories}
		\umlclass[x=36, y=10]{DocumentEntryFactory}{}{
		\umlstatic{+ instantiateComponent(document : Document*, type : ComponentDescriptor*, 
			position : QPointF,$\hookleftarrow$} \\
		\hspace{1cm} \umlstatic{id : QString, autoInstancePorts : bool) : DocumentEntry*} \\
		\umlstatic{+ instantiateInputPort(document : Document*, position : QPointF, id : QString) : DocumentEntry*} \\
		\umlstatic{+ instantiateOutputPort(document : Document*, position : QPointF, id : QString) : DocumentEntry*} \\
		\umlstatic{+ instantiateWire(document : Document*, sender : DocumentEntry*, 
			receiver : DocumentEntry*, $\hookleftarrow$} \\
 		\hspace{1cm} \umlstatic{id : QString) : DocumentEntry*} \\
		}
	\end{umlpackage}

	\umlVHVdep
	[anchor1= 130, anchor2=-50, pos stereo=1.5, stereo=generate]
	{DocumentEntryFactory}{DocumentEntry}
	
	\umlVHVdep[pos stereo= 1.5, stereo=use]{DocumentEntryFactory}{DocumentEntryType}
	\umlHVdep[pos stereo= 1.5, stereo=use, anchor2 = -50]{DocumentEntryFactory}{SchematicElement}
	
	\umlVHVdep
	[anchors= south and south, arm1 =-2cm, stereo=use, pos stereo=1.5]
	{DocumentEntryFactory}{ModelElement}

	\begin{umlpackage}{metamodel}
		\umlclass[x =46, y=4]{ComponentDescriptor}{}{}
	\end{umlpackage}

	\umlVHdep
	[anchor1=50, stereo=use, pos stereo=1.5]
	{DocumentEntryFactory}{ComponentDescriptor}

\end{umlpackage} % --- q2d

\end{tikzpicture} 
		}
		\caption{\gls{uml}-Diagram for the Document Entry and its Factory}
		\label{fig-class-diagram-documentEntry}
	\end{sidewaysfigure}		

	\begin{sidewaysfigure}
		\centering
		\resizebox{\textheight}{!}{
		\begin{tikzpicture}

\draw[draw=none](0,0) rectangle (52, 34);

\umlclass[x=26, y=32]{QObject}{}{
+ QObject(parent : QObject*)
}

\begin{umlpackage}{q2d}

\umlclass[x=13, y=28]{Document}{}{
+ model() : Model*
}
\umlVHinherit{Document}{QObject}

\umlclass[x=7, y=28]{DocumentEntry}{}{
+ model() : Model*
}
\umlVHinherit{DocumentEntry}{QObject}
\umlunicompo[mult1=1, mult2=*]{Document}{DocumentEntry}

\begin{umlpackage}{metamodel}

\umlclass[x=26, y=2]{ComponentDescriptor}{}{}

\end{umlpackage}


\begin{umlpackage}{model}

% --- Model
\umlclass[x=26, y=24]{Model}{}{
+ Model(parent : Document*) \\
+ addComponent(toAdd : Component*) \\
+ addConductor(toAdd : Conductor*) \\
+ addInputInterface(toAdd : ModuleInterface*) \\
+ addOutputInterface(toAdd : ModuleInterface*)
}
\umlinherit{Model}{QObject}
\umlVHassoc
[anchor1=130, arg1=model, mult1=1, arg2=parent, pos2=1.75, mult2=1]
{Model}{Document}

% --- ModelElement
\umlclass[x=36, y=18, type=abstract]{ModelElement}{}{
+ ModelElement(relatedEntry : DocumentEntry*) \\
+ localID() : QString \\
+  fullId() : QString \\
+ configVariables() : QStringList \\
+ inputVarialbles() : QStringList \\
+ nodeVariables() : QStringList \\
+ functions() : QStringList \\
+ propertyMap() : QMap<QString, QString>
}
\umlVHinherit{ModelElement}{QObject}
\umlHVassoc[anchor2=130, arg1=parent,  mult1=1, pos1=0.5, arg2=child, mult2=*, pos2=1.75]{Model}	{ModelElement}
\umlHVassoc[anchor1= 160, arg1=modelElement, mult1=1, arg2=relatedEntry, pos2=1.75, mult2=1]
	{ModelElement}{DocumentEntry}

% --- InterfacingME
\umlclass[x=10, y=18, type=abstract]{InterfacingME}{}{
+ addPort(port : Port*) \\
+ nodeVariables() : QStringList \\
}
\umlinherit{InterfacingME}{ModelElement}

% --- Node
\umlclass[x=36, y=12, type = abstract]{Node}{}{
+ Node(relatedEntry : DocumentEntry*) \\
+ addDriver(driver : ModelElement*) \\
+ addDrivenElement(drivenElement : ModelElement*)
}
\umlinherit{Node}{ModelElement}
\umlVHaggreg[anchors =150 and -130, arg2=driver, mult2=0..1, mult1=1, pos2=0.75]
	{Node}{ModelElement}
\umlVHaggreg[anchors =20 and -70, arg2=drivenElements, mult2=*, mult1=1, pos2=0.75]
	{Node}{ModelElement}

% -- Port
\umlclass[x=26, y=12, type=abstract]{Port}{
-direction : PortDirection
}{
+ Port(direction : PortDirection, $\hookleftarrow$ \\
	\hspace{1cm} relatedEntry : DocumentEntry*, $\hookleftarrow$ \\
	\hspace{1cm} interfaced : InterfacingME) \\
+ propertyMap() : QMap<QString, QString>
}
\umlinherit{Port}{Node}
\umlHVunicompo[anchor1 = -15, arg2=ports, mult1=1, mult2=*, pos2=1.75]{InterfacingME}{Port}

% --- ModulePort
\umlclass[x=22, y=8]{ModulePort}{}{
+ inputVariables() : QStringList \\
+ nodeVariables() : QStringList \\
}
\umlHVinherit{ModulePort}{Port}

% --- ComponentPort
\umlclass[x=30, y=8]{ComponentPort}{}{
+ nodeVariables() : QStringList \\
}
\umlHVinherit{ComponentPort}{Port}

% -- Component
\umlclass[x=6, y=6]{Component}{}{
+ Component(descriptor : ComponentDescriptor*, $\hookleftarrow$ \\
	\hspace{1cm} relatedEntry : DocumentEntry*) \\
+ configVariables() : QStringList \\
+ functions() : QStringList \\
}
\umlVHinherit[anchor1=130]{Component}{InterfacingME}
\umlVHunicompo[mult1=*, arg2=descriptor, mult2=1, pos2=1.8]
	{Component}{ComponentDescriptor}
\umlHVdep{Component}{ComponentPort}

% --- ModuleInterface
\umlclass[x= 11, y=12]{ModuleInterface}{
- direction : PortDirection
}{
+ ModuleInterface(relatedEntry : DocumentEntry*, $\hookleftarrow$ \\
	\hspace{1cm} moduleDirection : PortDirection)
}
\umlHVinherit{ModuleInterface}{InterfacingME}
\umlVHdep{ModuleInterface}{ModulePort}

% --- Conductor
\umlclass[x=46, y=18]{Conductor}{}{
+ Conductor(sender : Node*, receiver : Node*, $\hookleftarrow$ \\
	\hspace{1cm}relatedEntry : DocumentEntry*) \\
+ nodeVariables() : QStringList \\
+ functions() : QStringList \\
}
\umlinherit{Conductor}{ModelElement}
\umlVHunicompo[arg2=sender, mult2=1, mult1=1, pos2=1.8, anchors=-130 and 10]{Conductor}{Node}
\umlVHunicompo[arg2=receiver, mult2=1, mult1=1, pos2=1.8, anchors=-50 and -10]{Conductor}{Node}

\begin{umlpackage}{enums}
\umlclass[x=19, y=14, type=enum]{PortDirection}{
			IN \\
			OUT
		}{}
\end{umlpackage}
\umlHVdep[stereo=use]{Port}{PortDirection}
\umlHVdep[stereo=use]{ModuleInterface}{PortDirection}

\end{umlpackage}
\end{umlpackage}

\end{tikzpicture} 
		}
		\caption{\gls{uml}-Diagram for the \emph{q2d}-Model}
		\label{fig-class-diagram-model}
	\end{sidewaysfigure}

	\begin{sidewaysfigure}
		\centering
		\resizebox{\textheight}{!}{
		\begin{tikzpicture}

\draw[draw=none](0,0) rectangle (52, 34);

\umlclass[x=14, y=32]{QObject}{}{
+ QObject(parent : QObject*)
}

\umlclass[x=46, y=32]{QGraphicsObject}{}{}
\umlVHVinherit[arm1=0.5cm, anchors=north and north]{QGraphicsObject}{QObject}

\umlclass[x=36, y= 32]{QGraphicsItem}{}{}

\umlclass[x=26, y=32]{QGraphicsScene}{}{
+ addItem(item : QGraphicsItem)
}
\umlHVHinherit{QGraphicsScene}{QObject}
\umlinherit{QGraphicsObject}{QGraphicsItem}
\umlHVassoc[anchor2=150, arg1=scene,  mult1=1, pos1=0.25, arg2=item, mult2=*]
	{QGraphicsScene}{QGraphicsItem}

\begin{umlpackage}{q2d}

\umlclass[x=14, y=28]{Document}{}{
+ schematic() : Schematic*
}
\umlinherit{Document}{QObject}

\umlclass[x=7, y=28]{DocumentEntry}{}{
+ schematic() : Schematic*
}
\umlVHinherit{DocumentEntry}{QObject}
\umlunicompo[mult1=1, mult2=*]{Document}{DocumentEntry}

\begin{umlpackage}{metamodel}

\umlclass[x=26, y=2]{ComponentDescriptor}{}{}

\end{umlpackage}


\begin{umlpackage}{gui}

% --- Schematic
\umlclass[x=26, y=24]{Schematic}{}{
+ Schematic(parent : Document*)
}
\umlinherit{Schematic}{QGraphicsScene}
\umlVHassoc
[anchor1=130, arg1=schematic, mult1=1, arg2=parent, pos2=1.9, mult2=1]
{Schematic}{Document}

% --- SchematicElement
\umlclass[x=36, y=18, type=abstract]{SchematicElement}{}{
+ SchematicElement(position : QPointF, $\hookleftarrow$ \\
	\hspace{1cm}relatedEntry : DocumentEntry*) \\
+ addActual(actual : QGraphicsItem*) \\
+ scene() : Schematic* \\
\umlvirt{+ additionalJson() : QJsonObject} \\
\umlvirt{+ specificType() : QString}
}
\umlHVinherit{SchematicElement}{QGraphicsObject}
\umlHVassoc[anchor1= 160, arg1=schematicElement, mult1=1, arg2=relatedEntry, pos2=1.9, mult2=1]
	{SchematicElement}{DocumentEntry}
\umlVHVuniaggreg
	[mult1=1, mult2=*, arg2=actuals, pos2=2.75]
	{SchematicElement}{QGraphicsItem}
\umlHVdep
	[anchor2=130, mult1=1, mult2=*, pos2=1.75]
	{Schematic}{SchematicElement}

% --- ParentSchematicElement
\umlclass[x=10, y=18, type=abstract]{ParentSchematicElement}{}{}
\umlinherit{ParentSchematicElement}{SchematicElement}

% -- PortGraphicsItem
\umlclass[x=28, y=12]{PortGraphicsItem}{
- direction : PortDirection \\
- wireDrawingMode : bool
 }{
+ PortGraphicsItem(position : QPointF,  $\hookleftarrow$\\
	\hspace{1cm} relatedEntry : DocumentEntry*, direction : PortDirection)
}
\umlHVinherit[anchors = east and south]{PortGraphicsItem}{SchematicElement}
\umlHVdep[anchor1 = -15, pos1= 0.15, mult1=1, mult2=*, pos2=1.75]
	{ParentSchematicElement}{PortGraphicsItem}

% -- ComponentGraphicsItem
\umlclass[x=6, y=6]{ComponentGraphicsItem}{}{
+ ComponentGraphicsItem(position : QPointF, $\hookleftarrow$\\
	\hspace{1cm} relatedEntry : DocumentEntry*, $\hookleftarrow$ \\
	\hspace{1cm} descriptor : ComponentDescriptor) \\
+specificType() : QString \\
}
\umlVHinherit[anchor1=130]{ComponentGraphicsItem}{ParentSchematicElement}
\umlVHunicompo[mult1=*, arg2=descriptor, mult2=1, pos2=1.8]
 	{ComponentGraphicsItem}{ComponentDescriptor}

% --- ModuleInterfaceGI
\umlclass[x= 11, y=12]{ModuleInterfaceGI}{
- direction : PortDirection
}{
+ ModuleInterfaceGI(position : QPointF, $\hookleftarrow$\\
	\hspace{1cm} relatedEntry : DocumentEntry*, $\hookleftarrow$ \\
	\hspace{1cm} direction : PortDirection) \\
}
\umlHVinherit{ModuleInterfaceGI}{ParentSchematicElement}

% --- WireGraphicsItem
\umlclass[x=43, y=12]{WireGraphicsItem}{}{
+ WireGraphicsItem(start : PortGraphicsItem*, $\hookleftarrow$ \\
	\hspace{1cm} end : PortgraphicsItem*, $\hookleftarrow$ \\
	\hspace{1cm} relatedEntry : DocumentEntry*) \\
- routeLeftToRight() \\
- routeStraight() \\
+ route()
+ additionalJson() : QJsonObject
}
\umlHVinherit{WireGraphicsItem}{SchematicElement}
\umlVHVunicompo[arg2=start, mult2=1, mult1=1,  pos2=1.8, anchors=-140 and -50, arm2=-1cm]
	{WireGraphicsItem}{PortGraphicsItem}
\umlVHVunicompo[arg2=end, mult2=1, mult1=1, pos2=1.9, anchors=-130 and -70, arm2=-1.5]
	{WireGraphicsItem}{PortGraphicsItem}

% --- WireGraphicsLineItem
\umlclass[x=43, y= 6]{WireGraphicsLineItem}{}{
+ WireGraphicsLineItem(start : QPointF, end : QPointF $\hookleftarrow$ \\
	\hspace{1cm} parent : WireGraphicsItem*) \\
}
\umlcompo
	[mult1=1, mult2=1..*, arg2=actuals]
	{WireGraphicsItem}{WireGraphicsLineItem}

\begin{umlpackage}{enums}
\umlclass[x=19, y=14, type=enum]{PortDirection}{
			IN \\
			OUT
		}{}
\end{umlpackage}
\umlHVdep[stereo=use]{ModuleInterfaceGI}{PortDirection}
\umlHVdep[stereo=use]{PortGraphicsItem}{PortDirection}

\end{umlpackage}
\end{umlpackage}

\end{tikzpicture} 
		}
		\caption{\gls{uml}-Diagram for the \emph{q2d}-Schematic visualization}
		\label{fig-class-diagram-schematic}
	\end{sidewaysfigure}
	
		\begin{sidewaysfigure}
		\centering
		\resizebox{\textheight}{!}{
		\begin{tikzpicture}

\draw[draw=none] (0, 0) rectangle (52, 34);

\umlclass[x=26, y=30]{QObject}{}{}

\begin{umlpackage}{q2d}
\umlclass[x=23, y=26]{Document}{}{}

\begin{umlpackage}{model}
\umlclass[x=15, y=26]{Model}{}{}
\umlassoc[mult1=1, mult2=1, arg1=parent, arg2=model]
	{Document}{Model}

\umlclass[x=6, y=26]{ModelElement}{}{}
\umlassoc[mult1=*, mult2=1, arg1=child, arg2=parent]
	{ModelElement}{Model}

\end{umlpackage}

\begin{umlpackage}{quantor}

\umlclass[x=17, y=10, type=enum]{VariableType}{
+ CONFIG \\
+ INPUT \\
+ NODE
}{}

% --- QuantorInterface
\umlclass[x=26, y=17]{QuantorInterface}{
- solutions : QList<int>
}{
- buildContexts(contextSource : Model*, targetFunction : QString) \\
- interpreteSolution(result : Result) \\
+ slot\_solveProblem(targetDocument : Doczument*, targetFunction : QString) \\
+ signal\_hasSolution(textualRepresentation : QString, variableMapping : QMap<QString, bool>)
}
\umlinherit{QuantorInterface}{QObject}
\umlVHVdep[stereo=use, pos stereo=1.5, anchor1=130]{QuantorInterface}{Model}
\umlVHVdep[stereo=use, pos stereo=1.5, anchor1=130]{QuantorInterface}{Document}
\umlVHVdep[stereo=use, pos stereo=1.5, anchor1=south]{QuantorInterface}{VariableType}

% --- Context
\umlclass[type=abstract, x=8, y=8]{Context}{}{
+ operator[](name : std::string) : unsigned \\
+ typeOf(var : unsigned) : VariableType
}
\umlHVHdep[anchors=east and west, stereo=use, pos stereo=1.5]{Context}{VariableType}


% --- QIContext
\umlclass[x=8, y=17]{QIContext}{
- lowestIndex : unsigned \\
- highestIndex : unsigned \\
- variableMapping : QMap<QString, unsigned> \\
- typeMapping : QMap<unsigned, VariableType> \\
- functions : QList<std::string>
}{
- assignVariable(varName : QString, type VariableType) \\
+ QIContext(lowestIndex : unsigned, parent : QIContext*) \\
+ addModelElement(element : ModelElement*)
}
\umlinherit{QIContext}{Context}
\umluniaggreg
	[arg2=contexts, mult2=1..*, mult1=1]
	{QuantorInterface}{QIContext}
\umlHVHdep
	[arm1=-1cm, anchors= west and west, stereo= use, pos stereo=1.5]
	{QIContext}{ModelElement}

% --- Quantorizer
\umlclass[x=26, y=8]{Quantorizer}{}{
+ set(ctx : Context*) \\
+ parse(fct : char*) \\
+solve(ol : std::vector<int>\&) : Result
}
\umlVHVdep[anchors=south and south, arm1=-1cm, stereo=use, pos stereo=1.5]{Quantorizer}{Context}
\umldep[anchors=south and north, stereo=use]{QuantorInterface}{Quantorizer}

% --- ParseException
\umlclass[x=36, y=8]{ParseException}{}{
+ message() : std::string
}
\umldep[anchors=east and west, stereo=throw]{Quantorizer}{ParseException}

% --- Result
\umlclass[x=36, y=12]{Result}{}{
+ isSatisfiable() : bool \\
+ operator char*() : char*
}
\umlVHdep[anchors=70 and west, stereo=create, pos stereo=1.5]{Quantorizer}{Result}
\umlHVdep[anchors=east and north, stereo=use, pos stereo= 1.5]{QuantorInterface}{Result}


\end{umlpackage}
\end{umlpackage}

\end{tikzpicture}
 
		}
		\caption{\gls{uml}-Diagram for the \emph{q2d}-Quantor Interface}
		\label{fig-class-diagram-quantor}
	\end{sidewaysfigure}