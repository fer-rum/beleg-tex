\section{Forethoughts}

	\paragraph{Problem Statement}
	The aim of this work is the creation of a tool that can support hardware designers in evaluating the capabilities of \glspl{clb}.
	To achieve this, it is necessary to find ways of easily modelling configurable components using a graphical \gls{ui}.
	It is further required to offer a method to specify the intended behaviour of the whole design, at which point the tool should evaluate whether this specification can be implemented by the \gls{clb}, and by which configuration.
	

	\paragraph{Outline of the Solution}
	Initial research made clear that so far no tool exists for supporting this kind of \gls{clb} evaluation in a user friendly way.
	Attempts to extend already existing projects also proved to be difficult.
	As a consequence, a new application called \emph{q2d} was designed and implemented for the outlined purpose.
	It supports project management, \gls{clb}-based circuit design and the evaluation of such.
	The user is presented with an appealing graphical \gls{ui} and can create custom component types quickly, requiring only simple means.
	
	\paragraph{Involved Areas of Computer Science}
	In addition to the area of design and engineering of large-scale integrated circuits, several other fields of computer science are involved in creating an application capable of dealing with the posed task.
	As a result, the work at hand might also be worth reading for audiences engaged in these particular areas:
	
	\begin{description}
		\item[Computational Logic] is involved when discussing the theoretical background of the problem and solving the resulting \gls{qbf} problem in section \ref{sec-theoretical-background}.
		\item[\Gls{ui} Engineering] will be discussed largely in sections \ref{sec-ui-ergonomics}, \ref{sec-visualization-aspects} and \ref{sec-basic-interaction} where the presentation of information and user interaction will be discussed,
		\item[Software Engineering] comes in mostly during the design phase, when the choice of tools and the internal program structure is explained in sections \ref{sec-languages-frameworks}, \ref{sec-model-design} and its implementation in chapter \ref{sec-chapter-implementation}. 
	\end{description}

	\paragraph{Application Use Cases}
	The resulting application will operate on the register transfer level.
	It is designed to help the user focus on the design task he wishes to perform without distracting him with details regarding the concrete physical implementation of specific circuit elements.
	Therefore, \emph{q2d} is best suited to develop, test and explain concepts of configurable circuit designs and demonstrate their flexibility and limits.