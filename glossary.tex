%% glossary.tex
%%
%% Write down all your entries for your glossary here.
%% Abbreviations go to acronyms.tex
%%

%% --- C ---
\newdualentry{cbd}{CBD}{Component Behaviour Descriptor}{
The set of all \glspl{pcbd} referring to the same component context forms the holistic description of the components behaviour.
A formal definition can be found in section \ref{sec-definitions}.
}

%% --- D ---
\longnewglossaryentry{din}{name=DIN}{
	\emph{Deutsche Industrienorm}, German industrial standard.
}

%% --- G ---
\longnewglossaryentry{github}{name=GitHub}{
	A web-based provider for \emph{git}-repositories. The site features convenient tools for many standard tasks that come with the management of projects and repositories.
	The official website can be found at \url{www.github.com}.
}

%% --- P ---
\newdualentry{pcbd}{pCBD}{partial Component Behaviour Descriptor}{
A boolean formula, containing a subset of the node and configuration variables of a component context, is called a \emph{partial} \gls{cbd}.
All \glspl{pcbd} of a component form the \gls{cbd}.
See also section \ref{sec-definitions} and the entry for \gls{gls-cbd}.
}

%% --- Q ---

\longnewglossaryentry{qt}{name=Qt}{
A framework based on the \emph{C++} programming language. 
It extends the language features of \emph{C++} by the use of a meta-compiler and and thereby enables amongst other things the use of introspection and signal-slot concepts.
Common \glspl{os} like \emph{Linux}, \emph{Windows}, \emph{OS\,X} or \emph{Android} are well supported.
Furthermore, libraries are offered for a lot of standard tasks like the abstraction of graphical \glspl{ui}, \gls{json}-parsing, device interaction, unit testing\dots
\glspl{ide} specialized for the development with \emph{Qt} exist and provide support for the visual design of \glspl{ui} and internationalisation of developed applications while also integrating user-defined external tools and workflows.
The official website can be found at \url{qt-project.org}.
}

\longnewglossaryentry{quantor}{name=Quantor}{
An open-source solver for \gls{qbf} problems.
It was developed by the Johannes-Kepler University of Linz and is written in \emph{C}.
Internally, the \gls{sat} solver \emph{picosat} is employed.
Both tools can be found at \url{fmv.jku.at/quantor} and \url{fmv.jku.at/picosat} respectively.
}

\longnewglossaryentry{qucs}{name=Qucs}{
The \emph{Quite universal circuit simulator} is an open source tool for the visual design and simulation of analogous circuits.
Recent development has expanded it to offer basic design and evaluation features for digital components as well.
An exhaustive description can be found at \cite{Qucs:base} in combination with \cite{Qucs:devel}.
The official website can be found at \url{qucs.sourceforge.net}
}

\longnewglossaryentry{}{name=}{

}